\documentclass[
	,footlinenumber
	,navline=true
	,footlineauthor
	,ngerman
	]{beamer}

\definecolor{coolgray}{rgb}{0.90625,0.8828125,0.8359375}
\definecolor{BurntOrange}{rgb}{1,0.5,0}
\definecolor{Brown}{HTML}{C17D11}
\definecolor{SeaGreen}{HTML}{2e8b57}

\usepackage{babel}
\usepackage[utf8]{inputenc}
\usepackage[T1]{fontenc}
\usepackage{lipsum}
\usepackage{ifthen}
\usepackage{multirow}
\usepackage{tikz}
\usepackage{color}
\usepackage{soulutf8}
\usepackage{marvosym}
\usepackage{wasysym}
\usepackage{textcomp}
\usepackage{multicol}
\usepackage{etex}	%wegen mehr dimensions..
\usepackage{filecontents}
\usepackage{appendixnumberbeamer}
\usepackage{fancyvrb}

\usepackage{pgfpages}
%\setbeameroption{show notes on second screen=right}
\setbeamertemplate{note page}{%
	\insertnote%
}

\fvset{fontsize=\small}
\usepackage{listings}
\lstset{
  language=TeX,
  morekeywords={begin,documentclass,usepackage,section,subsection,subsubsection}, 
%Umlaute in listings 
  literate=%
	{Ö}{{\"O}}1
	{Ä}{{\"A}}1
	{Ü}{{\"U}}1
	{ß}{{\ss}}1
	{ü}{{\"u}}1
	{ä}{{\"a}}1
	{ö}{{\"o}}1,
	breaklines=true,
	}

	

\newcommand{\zwischenUebersicht}
{
   \begin{frame}
        \frametitle{\"Ubersicht}
	\note{
        \tableofcontents[ 
    		currentsubsection, 
%		hideothersections, 
		hideothersubsections,
   		sectionstyle=show/hide, 
%		sectionstyle=show/shaded, 
%		subsectionstyle=show/hide, 
		] 
	}
%	\tableofcontents[ 
%    		currentsubsection, 
%%		hideothersections, 
%		hideothersubsections,
%%   		sectionstyle=show/hide, 
%		sectionstyle=show/shaded, 
%		subsectionstyle=hide, 
%		] 
	\begin{columns}[t]
        \begin{column}{.5\textwidth}
            \tableofcontents[subsectionstyle=show,hideothersubsections,sectionstyle=show/shaded,sections={1-3}]
        \end{column}
        \begin{column}{.5\textwidth}
            \tableofcontents[subsectionstyle=show,hideothersubsections,sectionstyle=show/shaded,sections={4-}]
        \end{column}
    \end{columns}
   \end{frame}
}


%Easy Quoteblock
%Use: \quoteblock{Quoute Adii as manimus.}{Wurstlet}
\newcommand{\quoteblock}[2]
{
\begin{block}{}
\begin{quote}
    ``#1''
  \end{quote}
  \vskip3mm
  \hspace*\fill{\small--- #2} 
\end{block}
}

\newcommand{\miniblock}[2]
{
\hspace*{.1\linewidth}\begin{minipage}{.8\linewidth}
    \begin{block}{#1}
      #2
    \end{block}
    \end{minipage}
}

\usepackage{tikz}
\usetikzlibrary{shapes,snakes,arrows}
\usetikzlibrary{positioning}
\usetikzlibrary{arrows}
\usepackage{pgfplots}
\usetikzlibrary{calc,fadings,decorations.pathreplacing}

\usepackage{amsmath,amssymb}
\usepackage{array}
\usepackage{mathrsfs}
\usepackage{listings}

\usepackage{boxedminipage}

\graphicspath{{../pictures/}}

\tikzstyle{mybox} = [draw=red, fill=coolgray, very thick,
    rectangle, rounded corners, inner sep=5pt, inner ysep=10pt,
	text width=.9\textwidth]
\tikzstyle{fancytitle} = [fill=red, text=white, rounded corners]

\DeclareMathOperator{\ArcSin}{ArcSin}
\DeclareMathOperator{\ArcCos}{arc cos}

\mode<presentation>
{  
	\usetheme[headlinetitle={geht nicht, siehe outer theme z. 166}]{Bremen}
	\fbnum{03}
	\fbname{RY-AVS}
	\institutelogo{z. 159 im outertheme.}
	\title{geht nicht, siehe outer theme z. 166}
}

\usepackage[scaled=.90]{helvet}
\renewcommand\ttdefault{txtt}
\usefonttheme[onlymath]{serif}

\title{Protective Avionics Flash File System\\PAFFS}
\author[Pascal.Pieper@dlr.de]{Pascal Pieper}

\date{\today}


\begin{document}
\setbeamertemplate{caption}{\raggedright\insertcaption\par} %Kein ``Abbildung: `` vor captions


\begin{frame}
\titlepage
\end{frame}
\begin{frame}[t]{Inhaltsübersicht}
    \begin{columns}[t]
        \begin{column}{.5\textwidth}
            \tableofcontents[hideothersubsections,sections={1-3}]
        \end{column}
        \begin{column}{.5\textwidth}
            \tableofcontents[hideothersubsections,sections={4-6}]
        \end{column}
    \end{columns}
\note {
    \begin{multicols}{2}
    \tableofcontents
    \end{multicols}
    
    }
\end{frame}


\section{Einleitung}
\zwischenUebersicht
\subsection{Problemstellung}
\begin{frame}{Problemstellung}
\begin{block}{Computergestützte Raumfahrt}
\begin{itemize}
\item Besteht aus vielen Teilsystemen
\begin{itemize}
\item Fokus der Arbeit liegt auf Permanentspeicher
\end{itemize}
\item Messergebnisse 
\item Instruktionen 
\item Computerprogramme 
\begin{itemize}
\item[$\rightarrow$] Vieles muss zwischengespeichert werden
\end{itemize}

\end{itemize}

\note{
{Computergestützte Raumfahrt}
\begin{itemize}
	\item Besteht aus vielen Teilsystemen wie Sensoren, Aktuatoren, Rechenwerken
	\begin{itemize}
		\item Fokus der Arbeit liegt auf Permanentspeicher
	\end{itemize}
	\item Messergebnisse 
	\item Instruktionen 
	\item Computerprogramme 
	\begin{itemize}
		\item[$\rightarrow$] Vieles muss zwischengespeichert werden
	\end{itemize}
	
\end{itemize}	
}

\end{block}

\end{frame}

\begin{frame}{Problemstellung}
\begin{block}{Negative Einflüsse auf Speicher}
\begin{itemize}
\item Erschütterungen
\item Harte Strahlung
\item große Temperaturschwankungen
\item Temperaturabführung schwer
\end{itemize}
\end{block}

\note{
	{Negative Einflüsse auf Speicher}
		\begin{itemize}
			\item Erschütterungen
			\item Harte Strahlung
			\item große Temperaturschwankungen
			\item Temperaturabführung schwer, Darf nicht zu viel Wärme produzieren
		\end{itemize}
	}
\end{frame}

\begin{frame}{Problemstellung}
\begin{block}{Abhilfe}
\begin{itemize}
\item Strahlungsresistente oder robuste Speicher
	\begin{itemize}
	\item<alert@1-> Hohe Anschaffungskosten
	\end{itemize}
\end{itemize}
\end{block}
\begin{block}{Billige Speicher benutzen}
\begin{itemize}
\item Fehleranfälligkeit durch Dateisystem ausgleichen
\item Eine Logik optimiert Lebensdauer und Zuverlässigkeit
\end{itemize}
\end{block}

\end{frame}

\subsection{Idee}

\begin{frame}{Nutzen}
\begin{center}
Günstiger Speicher im Weltraum!
\note{Und leichter}
\end{center}
\end{frame}

\section{Anforderungen}
\begin{frame}{Anforderungen}
	\begin{block}{}
		\begin{itemize}
			\item Auf Eigenschaften eines NAND-Flashes eingehen
			\begin{itemize}
				\item[$\rightarrow$] Besonders auf die beschränkte Lebensdauer
			\end{itemize}
			\item Redundante Speicher parallel verwalten
			\item Bitfehler sowie Totalausfall einzelner Komponenten verkraften
			\item Minimalen RAM Verbrauch bei skalierbaren Flashgrößen aufweisen
			\item POSIX verwandte Dateiverwaltung ermöglichen
			\item Datenverlust bei spontanem Stromausfall vorbeugen
		\end{itemize}
	\end{block}
\end{frame}

\begin{frame}{Anforderungen}
	\begin{block}{Tradeoff}
		\begin{itemize}
			\item Lese-/Schreibgeschwindigkeit $\longleftrightarrow$ RAM Verbrauch
			\item Abnutzungsverringerung $\longleftrightarrow$ RAM Verbrauch, Ausfallsicherheit
			\item Speichereffizienz $\longleftrightarrow$ RAM Verbrauch, Ausfallsicherheit
		\end{itemize}
	\end{block}
\end{frame}


\section{Konzept}
\zwischenUebersicht
\subsection{Inodes und Tree Index}
\begin{frame}{Tree Index}
	\begin{block}{}
		\begin{itemize}
			\item a
		\end{itemize}
	\end{block}
\end{frame}

\subsection{Areas und Garbage Collection}
\begin{frame}{Areas}
	\begin{block}{}
		\begin{itemize}
			\item a
		\end{itemize}
	\end{block}
\end{frame}

\subsection{Superblock}
\begin{frame}{Superblock}
	\begin{block}{}
		\begin{itemize}
			\item a
		\end{itemize}
	\end{block}
\end{frame}

\subsection{Fehlerkorrektur und Redundanz}
\begin{frame}{Fehlerkorrektur}
	\begin{block}{}
		\begin{itemize}
			\item a
		\end{itemize}
	\end{block}
\end{frame}
\end{document}